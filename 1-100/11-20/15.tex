\documentclass{article} 

\begin{comment}
	tag:
	异常, 综合题。
\end{comment}


\begin{document}
	\paragraph*{Question 15} $\boxed{\text{Difficulty: Hard}} $			
	
	According to the C++17 standard, what is the output of this program?
	
	\begin{lstlisting}[language=C++]  		
#include <iostream>
#include <exception>

int x = 0;

class A {
	public:
	A() {
		std::cout << 'a';
		if (x++ == 0) {
			throw std::exception();
		}
	}
	~A() { std::cout << 'A'; }
};

class B {
	public:
	B() { std::cout << 'b'; }
	~B() { std::cout << 'B'; }
	A a;
};

void foo() { static B b; }

int main() {
	try {
		foo();
	}
	catch (std::exception &) {
		std::cout << 'c';
		foo();
	}
}
	\end{lstlisting}
	
	
	\paragraph*{答案和解析} $\boxed{\text{程序完全正确, 输出为: acabBA}} $
	
	这是一道综合题, 我们来分析它的流程。
	
	在主函数中, 调用 foo 后, 我们尝试初始化 b。在调用构造函数 B() 之前, 还要先构造成员函数 a。
	
	在调用 A() 的过程中, 输出 a 并引发异常。注意, 此时 a 的构造还没有完成, 因此之后也不会调用 a 的析构函数(相对地, 如果 a 有一些已经构造的其他成员, 退出时依然会析构它们)。
	
	catch(std::exception \&) 这里捕获了异常, 输出 c 后再次调用 foo()。这一次将顺利构造, 按顺序输出 ab, 程序结束后按相反顺序输出BA。
	
\end{document}
