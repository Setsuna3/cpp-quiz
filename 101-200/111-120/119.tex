\documentclass{article}

\begin{comment}
	tag:
	变量定义和初始化
\end{comment}

\begin{document}
	\paragraph*{Question 119}\noindent $\boxed{\text{Difficulty: Medium}} $
	
	According to the C++17 standard, what is the output of this program? 
	
	\begin{lstlisting}[language=C++]    
#include <iostream>

int main() {
	void * p = &p;
	std::cout << bool(p);
}
	\end{lstlisting}
	
	\paragraph*{答案和解析} $\boxed{\text{程序完全正确,输出为:1}} $
	
	\href{https://timsong-cpp.github.io/cppwp/n4659/basic.scope.pdecl#1}{[basic.scope.pdecl\#1]} 指出, 变量名定义的时间点在完整声明者(declarator)之后, 在初始化者(initializer)之前。
	\begin{lightgrayleftbar}
		The point of declaration for a name is immediately after its complete declarator and before its initializer (if any)
	\end{lightgrayleftbar}
	这里的完整声明者是第一个 p, 初始化者是 \&p, 在两者之间完成了变量名的定义, 因此初始化者可以使用 p 并且取得它的地址。第 4 行语句结束后, p 是一个指向自己的指针, 它一定不是空指针, 因此答案是 1。
	
	这里再给出几个练习。
	请回答, 在 C++17 标准下, 以下程序的答案。
	
	\begin{lstlisting}[language=C]    
#include <iostream>
int main()
{
	const int x = 42;
	{int x = x ; std::cout << x; }
}
	\end{lstlisting}

	A: 是42。
	
	B: 有可能不是42。
	
	C: 无法编译。
	
	D: ub, 取决于具体实现。
	
	\begin{lstlisting}[language=C++]    
#include <iostream>
int main()
{
	const int x = 42;
	{int x[x] ; std::cout << sizeof x; }
}
	\end{lstlisting}

	在这个问题中, 假设 sizeof(int) = 4
	
	A: 是168。
	
	B: 有可能不是168。
	
	C: 无法编译。
	
	D: ub, 取决于具体实现。
	
	参考答案: 
	
	第一个程序中, x 在初始化之前已经完成了声明, 因此初始化将使用它本身(尚未确定)的值, 所以可能不是42。
	
	第二个程序中, 直到 x[x] 处声明者才完全结束, 声明前还没有隐藏外部变量 x, 因此这里数组大小 x 是外部的 x, 这里等效于声明 int x[42], 答案是168。
	
\end{document}
