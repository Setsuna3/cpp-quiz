\documentclass{article}
\begin{document}
	\paragraph*{Question 226}\noindent $\boxed{\text{Difficulty: Medium}} $
	
	According to the C++17 standard, what is the output of this program? 
	
	\begin{lstlisting}[language=C++]    
#include <iostream>
#include <utility>

struct X {
	X() { std::cout << "1"; }
	X(X &) { std::cout << "2"; }
	X(const X &) { std::cout << "3"; }
	X(X &&) { std::cout << "4"; }
	~X() { std::cout << "5"; }
};

struct Y {
	mutable X x;
	Y() = default;
	Y(const Y &) = default;
};

int main() {
	Y y1;
	Y y2 = std::move(y1);
}
	\end{lstlisting}
	
	\paragraph*{答案和解析} $\boxed{\text{程序完全正确,输出为:1255}} $
	
	主函数的第一行定义了 Y 类型的变量 y1,因此调用 Y 的构造函数。
	
	类型 Y 拥有一个 X 类型的成员 x,因此调用 x 的默认构造函数,输出 1。
	
	第二行中,std::move 将 y1 变换为右值,但是 Y 并没有移动构造函数,只能退而求其次,使用复制构造函数。因此, 其成员 x 也将使用 X 类的复制构造函数进行构造。
	
	接下来的问题就是,我们将采用 X 类的哪一个构造函数。此时,x 本应该是 const X\& 类型的,但有 mutable 修饰,根据 \href{https://timsong-cpp.github.io/cppwp/n4659/dcl.stc#9}{dcl.stc\#9}
	
	\begin{lightgrayleftbar}
		The mutable specifier on a class data member nullifies a const specifier applied to the containing class object and permits modification of the mutable class member even though the rest of the object is const.
	\end{lightgrayleftbar}
	
	因此, x 并不是 const 的, 在两个复制构造函数中,选择第六行的 X(X \&) 较为合适, 故输出 2。
	
	在 main 函数结束后, y1 和 y2 执行析构函数。输出两次 5。
\end{document}
