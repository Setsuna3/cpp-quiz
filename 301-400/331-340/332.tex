\documentclass{article}
\begin{document}
	\paragraph*{Question 232} $\boxed{\text{Difficulty: Hard}} $			
	
	According to the C++17 standard, what is the output of this program?
	
	\begin{lstlisting}[language=C++]  		
#include <iostream>

struct S {
	template <typename Callable>
	void operator[](Callable f) {
		f();
	}
};

int main() {
	auto caller = S{};
	caller[ []{ std::cout << "C";} ];
}
	\end{lstlisting}
	
	\paragraph*{答案和解析} $\boxed{\text{程序将引发编译错误}} $
	
	事实上, 程序逻辑没有任何问题, 但是如果尝试执行, 编译器(gcc)会提示
	
	error: two consecutive '[' shall only introduce an attribute before '[' token.
	
	查阅标准\href{https://timsong-cpp.github.io/cppwp/n4659/dcl.attr.grammar#7}{[dcl.attr.grammar\#7]}
	
	\begin{lightgrayleftbar}
		Two consecutive left square bracket tokens shall appear only when introducing an attribute-specifier or within the balanced-token-seq of an attribute-argument-clause.
	\end{lightgrayleftbar}

	这是在说, 一般只有类似 \verb|[[nodiscard]] node* f(int x) {return new node(x);}| 之类的情况, 才用到两个连续的左中括号。对于这题而言, 我们不去深究 balanced-token-seq of an attribute-argument-clause 等概念的具体含义, 因为标准给出了一个相似的例子。
	
	\begin{lstlisting}[language=C++]
int p[10];
int(p[[x] { return x; }()]);  // error: invalid attribute on a nested declarator-id and not a function-style cast of an element of p.
	\end{lstlisting}

\end{document}
