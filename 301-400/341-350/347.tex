\documentclass{article} 
\begin{comment}
	tag:
	std::string
\end{comment}
\begin{document}
	
	\paragraph*{Question 347} $\boxed{\text{Difficulty: Medium}} $			
	
	According to the C++17 standard, what is the output of this program?
	
	\begin{lstlisting}[language=C++]  		
#include <iostream>
#include <type_traits>

template <typename T>
void foo(T& x)
{
	std::cout << std::is_same_v<const int, T>;
}

template <typename T>
void bar(const T& x)
{
	std::cout << std::is_same_v<const int, T>;
}

int main()
{
	const int i{};
	int j{};
	
	foo(i);
	foo(j);
	bar(i);
	bar(j);
	
}
	\end{lstlisting}
	
	
	\paragraph*{答案和解析} $\boxed{\text{程序正常运行, 输出为: 1000}} $
	
	在进行函数模板的参数推导时, 可以参照 \href{https://timsong-cpp.github.io/cppwp/n4659/temp.deduct.call#3}{temp.deduct.call\#3} 中的规则。
	
	\begin{lightgrayleftbar}
		If P is a cv-qualified type, the top-level cv-qualifiers of P's type are ignored for type deduction. If P is a reference type, the type referred to by P is used for type deduction.
	\end{lightgrayleftbar}
	
	即, 进行模板参数推导时, 不考虑函数形参的顶级 cv 限定符及引用。
	
	foo 函数的参数类型可表为 T, 而 i 的类型是 const int, 因此 foo(i) 中, T 被推导为 const int, 输出 1。
	
	j 的类型是 int, 因此 foo(j) 中 T = int, 输出 0。
	
	bar 函数的参数类型可表为 const T, 而 i 的类型是 const int, 因此 bar(i) 中, T 被推导为 int, 输出 0。
	
	j 的类型是 int, 无法与 const T 匹配。但此时根据 \href{https://timsong-cpp.github.io/cppwp/n4659/temp.deduct.call#4}{temp.deduct.call\#4} 中的规则, 模板匹配一般要求类型完全一致, 但有例外。
	
	\begin{lightgrayleftbar}
		If the original P is a reference type, the deduced A (i.e., the type referred to by the reference) can be more cv-qualified than the transformed A. 
	\end{lightgrayleftbar}
	
	即, 若形参是引用类型, 则允许通过推导, 使得实参类型转化为更加 符合 cv 条件(cv-qualified)的类型。这里将 T 推导为 int, 则 实参类型 const int 比原先类型 int 更加符合 cv 条件, 这一推导是允许的。
	
	因此, T 被推导为 int, 输出 0。
\end{document}