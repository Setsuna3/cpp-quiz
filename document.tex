\documentclass{article} 

% 导入中文宏
\usepackage{ctex}
\usepackage{framed} 
\usepackage{fancyhdr} % 设置页眉、页脚
\usepackage{amsmath}
\usepackage{graphicx}
\usepackage{listings} % 代码
\usepackage{comment}
\usepackage[table,xcdraw]{xcolor}
\usepackage{booktabs}
\usepackage{hyperref}
\usepackage{framed}
\usepackage{subfiles}

\setcounter{secnumdepth}{4}
\setcounter{tocdepth}{4}

\pagestyle{fancy}
\fancyhf{}

\usepackage{color}

\chead{C++ Quiz Tutorial}%页眉中间

\usepackage{color}

\lstset{ %基本设置
	basicstyle=\small, %环境中的代码字体变小 [在结尾要加逗号!]
	basicstyle=\tt, %使用teletype字体(一种等宽字体)
	basicstyle=\it, %使用罗马斜体
	%其他可选的还有:\bf, \sc, \st, \sl, \em, \nocorr等
}
\lstset{ %关键字设置
	keywordstyle=\color{black}, %设置关键字颜色为黑色
	keywordstyle=\color{blue}, %设置关键字颜色为蓝色
	keywordstyle=\bfseries, %不改变当前字体的族与形状,但转变成bold加粗序列
	keywordstyle=\mdseries, %不改变当前字体的族与形状,但转变成中等粗细medium序列
	keywordstyle=\underbar, %添加下划线
	keywordstyle=\color{black}\bfseries\underbar, %添加下划线的加粗黑色关键字
}
\lstset{
	breaklines, %自动换行
	identifierstyle=\color{red},
	%identifierstyle= , %不设置 
}

\lstset{ %注释设置
	commentstyle=\it\color[RGB]{100,100,100}, 
}

\lstset{
	basicstyle=\tt,
	keywordstyle=\color{purple}\bfseries,
	identifierstyle=\color{brown!80!black},
	commentstyle=\color{gray}
	%showstringspaces=false,
}

\lstset{
	numbers=left, %在左侧显示行数
	numberstyle=\tiny\color{black}, %数字大小,颜色调整
	stepnumber=1 , %每两行标号一次
	numbersep=5pt, %数字右端(若为左侧显示数字)水平距离代码5pt
} 

\lstset{
	tabsize=4,
	xleftmargin=2em, %整体距左侧边线的距离为2em
	xrightmargin=2em,
}

\lstset{
	frame=shadowbox, %设置阴影
	framexleftmargin=4mm, %框架左边界延长(frame是不会考虑到左边的行数栏的)
	rulecolor=\color{gray}, %框架颜色设置
	rulesepcolor=\color{gray}, %框架阴影颜色设置
}

\definecolor{lightgray}{rgb}{0.75,0.75,0.75} 
\newenvironment{lightgrayleftbar}{
	 \def\FrameCommand{\textcolor{lightgray}{\vrule width 3pt} \hspace{3pt}} \MakeFramed {\advance\hsize-\width \FrameRestore}}
{\endMakeFramed}
 
\cfoot{第 \thepage 页, 共 \pageref{unknown} 页} 
\renewcommand{\footrulewidth}{0.1mm} 

\begin{document}
	
	\tableofcontents
	
	\subfile{introduction}
	% section 1 
	\section{C++ 基本问题}

	\subsection{C++ 基本类型}
	\subfile{tutorials/101-200/151-160/151} % ok [E] char 的符号性、位数都是未指定的。
	
	\subfile{tutorials/201-300/271-280/275} % ok
	
	\subsection{C++ 基本语法}
	
	\subfile{tutorials/1-100/1-10/6} % ok [E] for 循环
	
	\subfile{tutorials/101-200/111-120/111} % ok [E] continue 语句
	
	\subfile{tutorials/301-400/331-340/333} % ok [M] 快速趋向于!
	
	\subfile{tutorials/101-200/161-170/161} % ok [H] switch 
	
	\subsection{字面量的类型}
	\subfile{tutorials/101-200/111-120/115} % ok [E] 基础的函数重载匹配, 问浮点字面量是什么类型
	
	\subfile{tutorials/1-100/1-10/4} % ok [E] 基础的函数重载匹配, 问浮点字面量是什么类型
	
	\subfile{tutorials/301-400/331-340/335} % ok [M] 字面量NULL的类型问题
	
	\subfile{tutorials/201-300/271-280/277} % ok
	
	\subsection{变量的作用域和生命周期}
	
	\subfile{tutorials/101-200/191-200/197} % ok [E] 当作用域内的变量和作用域外的变量同名。
	\subfile{tutorials/101-200/101-110/105} % ok [M] 变量的生命周期
	
	\subfile{tutorials/201-300/291-300/295} % ok [E] 临时变量析构时间, copy elision
	
	\subsection{变量的声明和定义}
	
	\subfile{tutorials/201-300/291-300/291} % ok [M] 变量名
	
	\subfile{tutorials/101-200/111-120/119} % ok [M] 变量声明时间点
	
	\subfile{tutorials/101-200/191-200/198} % ok [M] extern 变量声明
	
	\subfile{tutorials/101-200/101-110/106} % ok [M] extern 变量声明
	
	\subsection{变量的初始化}
	\subfile{tutorials/1-100/11-20/11} % ok [E] 全局变量初始化
	
	\subfile{tutorials/1-100/11-20/12} % ok [E] 静态变量初始化
	
	\subfile{tutorials/201-300/291-300/296} % ok [M] 聚合体初始化 
	\subfile{tutorials/101-200/101-110/107} % ok [M] 初始化列表评估顺序
	\subfile{tutorials/201-300/221-230/221} % ok [E] 大括号初始化/direct initialization
	
	\subsection{cv 限定符}
	\subfile{tutorials/101-200/111-120/114} % ok 
	\subfile{tutorials/101-200/141-150/148} % ok valitaile
	
	\subsection{函数类型}
	\subfile{tutorials/201-300/231-240/233} %todo
	
	\subsection{习题}
	\subfile{exercises/basic/1}
	\subfile{exercises/basic/2}
	\subfile{exercises/basic/3}
	\subfile{exercises/basic/4}
	\subfile{exercises/basic/5}
	\subfile{exercises/basic/6}
	
	\section{C++ 表达式和类型转换}
	
	\subsection{运算和运算符}
	
	\subfile{tutorials/1-100/21-30/26} % ok [E] 除以零
	
	\subfile{tutorials/201-300/211-220/220} % ok [E] ?: 三元运算
	\subfile{tutorials/101-200/111-120/120} % ok [E] 运算符优先级问题
	
	\subfile{tutorials/201-300/261-270/265} % ok [E] 基础的函数重载匹配, 问&x是左值还是右值
	
	\subfile{tutorials/201-300/281-290/286} % ok [M] integral promotion, 转换。
	
	\subfile{tutorials/201-300/251-260/259} % ok [M] char, integral promotion
	
	\subfile{tutorials/1-100/21-30/25} %ok
	
	\subfile{tutorials/101-200/141-150/144} % ok [H] 无符号数取负数
	\subfile{tutorials/1-100/21-30/24} % ok [E] 无符号数运算
	
	\subfile{tutorials/201-300/281-290/288} % ok [E] 
	\subfile{tutorials/201-300/211-220/217} % ok [H] ?: 三元运算
	\subfile{tutorials/201-300/231-240/238} % ok [M] 综合
	
	\subfile{tutorials/101-200/171-180/178} % 词法分析
	\subsection{类型转换}
	
	\subfile{tutorials/101-200/151-160/153} % ok [M] cv-qualifed, const char* 和 char*的转换。
	
	\subfile{tutorials/201-300/241-250/249} %todo

	\subfile{tutorials/101-200/181-190/190} % ok
	\subfile{tutorials/201-300/201-210/205} % narrowing_cast
	
	\subfile{tutorials/201-300/231-240/236} % [M] 类型转换模板

	\subsection{cast 类型转换}
	 
	\subfile{tutorials/101-200/171-180/179} %todo
	\subfile{tutorials/101-200/181-190/188} % ok [M] 字符串字面量不能修改
	\subsection{习题}
	\subfile{exercises/expr/1}
	\subfile{exercises/expr/2}
	
	\section{C++ 函数}
		\subsection{函数声明与函数调用}
		\subfile{tutorials/1-100/21-30/30} % ok [E] 函数与变量的二义性
		\subfile{tutorials/1-100/31-40/31} % ok [H] 函数与变量的二义性
		
		\subfile{tutorials/1-100/1-10/9} % ok [E] 引用作为函数参数
		\subfile{tutorials/101-200/181-190/186} % ok [M] 数组作为函数参数
		\subfile{tutorials/101-200/131-140/132} % ok [E] 函数默认值
		\subfile{tutorials/201-300/281-290/289} % ok [M] 函数默认值
		\subfile{tutorials/101-200/131-140/140} % ok [M] 数组作为函数参数
		
		\subfile{tutorials/301-400/301-310/305} % ok [M] 函数参数评估顺序
		\subfile{tutorials/101-200/191-200/192} % ok [M] 函数参数评估顺序
		\subfile{tutorials/101-200/151-260/159} % ok [M] 函数参数评估和函数体顺序
		
		\subfile{tutorials/201-300/221-230/227} %todo
		\subsection{名称查找}
		\subfile{tutorials/101-200/121-130/126} % ok [E] 静态成员定义时的名称查找
		\subfile{tutorials/101-200/181-190/184} % ok [M] 类内进行名称查找的情况
		\subfile{tutorials/101-200/191-200/196} % todo [M] namespace 和名称查找
		\subsection{函数重载}
		
		
		\subfile{tutorials/101-200/111-120/118} % ok [E] 重载
		\subfile{tutorials/1-100/1-10/3} % ok [E] 重载

		
		\subfile{tutorials/1-100/1-10/2} % ok [M] 重载
		
		\subsection{习题}
		\subfile{exercises/fun/1} 
		\subfile{exercises/fun/2}
	\section{C++ 重要关键字}
	\subsection{sizeof}
	\subsection{auto}
	\subfile{tutorials/201-300/271-280/276} % ok [H] 递归推导
	\subfile{tutorials/101-200/161-170/163} % todo [E] 引用
	\subsection{typeid}
	\subsection{typedef}
	\subsection{decltype}
	\subfile{tutorials/1-100/31-40/37} % ok [E] decltype 规则
	\subfile{tutorials/1-100/31-40/38} % ok [M] decltype 规则
	\subfile{tutorials/301-400/331-340/337} % todo
	
	\section{C++ 类和对象}
	
	\subsection{构造函数和初始化}
	
	\subfile{tutorials/1-100/21-30/28} % todo
	\subfile{tutorials/1-100/31-40/32} % todo
	\subfile{tutorials/201-300/261-270/264} % todo
	\subfile{tutorials/101-200/181-190/187} % [E] 复制构造函数
	\subfile{tutorials/201-300/221-230/226} % [M] 构造函数的选择
	\subfile{tutorials/201-300/281-290/281} % todo
	
	\subsection{构造函数、析构函数的调用顺序}
	
	\subfile{tutorials/1-100/1-10/5} % ok [E] 成员初始化顺序
	
	\subfile{tutorials/201-300/281-290/283} % ok [M] 构造和delete析构的顺序
	
	\subfile{tutorials/1-100/11-20/13} % ok [E] 全局变量的构造和析构
	
	\subfile{tutorials/1-100/11-20/16} % ok [E] 成员构造和构造函数的顺序
	
	\subfile{tutorials/101-200/141-150/145} % todo
	
	\subfile{tutorials/1-100/11-20/14} % todo
	
	\subfile{tutorials/1-100/11-20/17} % todo
	
	\subsection{类的继承和多态}
	
	\subfile{tutorials/301-400/311-320/312} % [E] private 继承
	\subfile{tutorials/1-100/21-30/29} % ok [M] 在构造和析构的时候调用虚函数
	 
	\subfile{tutorials/1-100/1-10/7} % ok [E] 虚函数
	\subfile{tutorials/1-100/1-10/8} % ok [E] 虚函数
	\subfile{tutorials/1-100/11-20/18} % ok [E] 虚函数和访问控制
	\subfile{tutorials/1-100/31-40/33} % ok [E] 虚函数
	\subfile{tutorials/1-100/21-30/27} % ok [E] 虚函数
	\subfile{tutorials/1-100/41-50/44} %todo
	\subfile{tutorials/101-200/151-160/160} % ok [M] 虚函数和默认值

	\subsection{友元函数}
	
	\subfile{tutorials/1-100/51-60/52} % 友元函数的定义
	
	\subsection{习题}
	\subfile{exercises/class/1}
	\section{C++ 模板}
	
	\subsection{模板参数推导}
	
	\subfile{tutorials/101-200/181-190/185} % ok [M] 模板 T 推导
	\subfile{tutorials/101-200/121-130/125} % ok [M] 模板特化
	\subfile{tutorials/301-400/341-350/347} % ok [M] 带 cv 和 引用的类型推导
	
	\subsection{模板的定义和初始化}
	\subfile{tutorials/201-300/241-250/243} % todo
	\subfile{tutorials/301-400/331-340/338} % todo
	\subfile{tutorials/1-100/1-10/1} % todo [M] 模板特化时的函数推导
	\subfile{tutorials/101-200/111-120/113} % ok []
	\subfile{tutorials/101-200/101-110/109} % todo
	\subfile{tutorials/201-300/241-250/242} % todo
	\subfile{tutorials/201-300/241-250/250} % todo
	
	\subsection{模板与函数重载}
	\subfile{tutorials/201-300/251-260/251} % todo
	
	\subsection{模板与名称查找}
	\subfile{tutorials/101-200/161-170/162} %[M] 模板继承时的名称查找
	
	\section{C++ 容器}
	
	\subsection{string}
	\subfile{tutorials/201-300/281-290/287} % ok std::string 构造函数
	\subfile{tutorials/201-300/281-290/284} % ok [M] std::string 是否以\0结尾?
	
	\subsection{initializer\_list}
	\subfile{tutorials/201-300/231-240/235} % ok [M] std::initializer_list 作为参数
	
	\subsection{vector}
	
	\subfile{tutorials/1-100/31-40/35} % ok [E] std::vector 构造函数
	
	\subsection{tuple}
	
	\subfile{tutorials/201-300/271-280/278} % ok [E] std::tuple<T>
	\subsection{map}
	\subfile{tutorials/201-300/201-210/208} % ok [E] std::map operator[]
	\subfile{tutorials/101-200/131-140/135} % ok [E] std::map 构造
	


	\section{C++ 异常处理}
	\subfile{tutorials/201-300/231-240/239} % ok [E] 异常匹配
	\subfile{tutorials/1-100/11-20/15} % ok [H] 综合题 
	\subfile{tutorials/301-400/321-330/323} % todo 综合题
	\section{C++ 新特性}
	\subsection{lambda 表达式}
	\subfile{tutorials/201-300/221-230/229} % todo
	\subsection{折叠表达式}
	\subfile{tutorials/201-300/211-220/219} % todo
	\subfile{tutorials/301-400/311-320/313} % todo
	\subsection{std::variant}
	
	\subfile{tutorials/201-300/271-280/279} % ok std::variant 和 std::visit
	\subfile{tutorials/201-300/221-230/222} % ok std::variant 构造函数
	
	\subsection{promise/future}
	
	\subfile{tutorials/301-400/331-340/339} % ok
	\subfile{tutorials/301-400/331-340/340} % ok

	
	\section{杂项}
	\subfile{tutorials/101-200/191-200/193} % ok
	\subfile{tutorials/101-200/191-200/195} % ok
	\subfile{tutorials/201-300/291-300/293} % ok
	\subfile{tutorials/201-300/221-230/224} % ok
	\subfile{tutorials/301-400/331-340/332} % ok
	\subfile{tutorials/101-200/141-150/147} % ok trigraph
	\subfile{tutorials/1-100/41-50/41} %ok
	
	\section{习题参考答案}
	\subsection{C++ 基本问题}
	\subfile{answers/basic/1}
	\subfile{answers/basic/2}
	\subfile{answers/basic/3}
	\subsection{C++ 函数}
	\subfile{answers/fun/1}
	\subfile{answers/fun/2}
	\subsection{C++ 类和对象}
	\subfile{answers/class/1}
	
	\subfile{Epilogue}
\label{unknown}
\end{document}