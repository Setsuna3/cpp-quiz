\documentclass{article} 

\usepackage{ctex}
\usepackage{framed} 
\usepackage{fancyhdr} % 设置页眉、页脚
\usepackage{amsmath}
\usepackage{graphicx}
\usepackage{listings} % 代码
\usepackage{comment}
\usepackage[table,xcdraw]{xcolor}
\usepackage{booktabs}
\usepackage{hyperref}
\usepackage{framed}

\begin{document}
\section{前言}
这是一份 pzr 在 \href{https://cppquiz.org}{cpp quiz} 网站上的刷题笔记, 其中的题目形式如下。

\begin{itemize}
	\item 给出一段代码,请说出代码在 	\href{https://www.iso.org/standard/68564.html}{ISO C++17}
	标准下的运行结果。
\end{itemize}

给出的代码不一定是完全正确的,在代码不正确的情况,你需要指出它是以下哪种情况:

\begin{itemize}
	\item 代码中无法通过编译。
	\item 代码存在未指定行为(unspecified behavior)或实现定义(implementation defined)行为。
	\item 代码中存在未定义行为(undefined behavior)。
\end{itemize}

每道题目将附有解析, 解析中对于不是显见的结论, 将给出在 \href{https://github.com/cplusplus/draft/blob/main/papers/n4659.pdf}{N4659} 中的引用。

需要指出的是, 这篇笔记并不是针对 C++ 学习的, 这些题目对于 C++ 的学习也并不一定都是有意义的。但是, 刷题的过程还是相当有意思的。感兴趣的同学, 可以在 \href{https://cppquiz.org}{cpp quiz} 网站上进行尝试。

注:
\begin{itemize}
	\item (1)编译错误:程序中存在C++标准所不允许的行为, 不应该通过编译。例如使用 C++关键字(keyword) 作为变量名。
	\item (2)未定义行为:标准没有做任何要求的行为, 结果是不可预测的。例如除以 0。如果程序是 ill-formed, 我们认为是具有未定义行为的, 而不是编译错误。
	\item (3)未指定行为:标准通常给出了可能的实现范围, 具体的选择由实现决定。例如函数参数的评估顺序。
	\item (4)实现定义行为:结果取决于具体实现和实现文档, 例如sizeof(int)的值。
\end{itemize}
\end{document}